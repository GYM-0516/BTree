\documentclass[12pt]{article}
\usepackage{ctex}
\usepackage{amsmath} 
\usepackage{amssymb} 
\usepackage{enumerate}
\usepackage{clrscode3e}
\usepackage{algorithm}
\usepackage{algorithmicx}
\usepackage{algpseudocode}
\usepackage{geometry}
\usepackage{lmodern}
\usepackage{amssymb,amsmath}
\usepackage{ifxetex,ifluatex}
\usepackage{fixltx2e} % provides \textsubscript
\usepackage{listings} 
\usepackage{graphicx}
\geometry{a4paper,left=2cm,right=2cm,top=1cm,bottom=1cm}
%\newcommand{\song}{\CJKfamily{song}} 
%\newcommand{\xiaosihao}{\fontsize{15pt}{\baselineskip}\selectfont}
\begin{document}
	\newpage
	\section{B-Tree的定义}
	一颗B\_Tree是一个具有下列性质的有根数,假设根结点是T.root:
	\begin{enumerate}
		\item 每个结点的性质
		\begin{enumerate}
			\item $x.n$ 当前结点中关键字的个数
			\item $x.n$个关键子本身 $x.key_1,\ x.key_2,\ \cdots,\ x.key_{x.n}$非降序排列,使得$x.key_1 \leqslant x.key_2 \leqslant \cdots \leqslant x.key_{x.n}$
			\item  $x.leaf$是一个$bool$至。如果$x$是叶子结点,则为$TRUE$,否则为$FALSE$
		\end{enumerate}
		\item 每个内部结点有$x.n+1$个指向孩子的指针$x.c_1,\ x.c_2,\ \cdots,\ x.c_n$。叶子结点没有孩子,它们的孩子指针无定义。
		\item 关键字$x.key_i$对存储在各子树的关键字加以分割:如果$k_i$是任意一个存储在以$x.c_i$为根的子树中的关键字,那么
		\[k_1\leqslant x.key_1\leqslant k_2\leqslant x.key_2\leqslant\cdots\leqslant x.key_{x.n}\leqslant k_{x.n+1}\]
		\item 每个叶子结点具有相同的深度,即树高$h$
		\item 每个结点的关键字个数有上界和下界。用最小度数的固定整数$t\geqslant 2$表示这些上下界:
		\begin{enumerate}
			\item 除根结点之外,每个结点至少有$t-1$个关键字和$t$个孩子。非空树的情况下,根结点至少有一个关键字。
			\item 每个结点最多有$2t-1$个关键字和$2t$个孩子。当一个结点有$2t$个孩子时,该结点是满的。
		\end{enumerate}
	\end{enumerate}
	\section{B-Tree的高度}
	如果$n\geqslant 1$,对于任意一颗包含$n$个关键字、高度是$h$、最小度数$t\geqslant 2$的B\_Tree,都有:
	\[h\leqslant \log_{t}{\dfrac{n+1}{2}}\]
	
	\section{B-Tree查找关键字}
	B树寻找关键字的过程类似于二叉搜索树,区别在于二叉搜索树在每个结点只需要确定两个方向,而B树需要确定多个方向.查找过程采用递归的方式,这里使用尾递归.易于理解,编译优化后效率较高.B树搜索过程需要进行$O(\log_tn)$次磁盘的读取,总的时间复杂度为$O(t\log_tn)$
	   \begin{codebox}
	 	\Procname{$\proc{B-TREE-SESRCH}(x,k)$}
	 	\li $i=1$
	 	\li \While \ $i\leqslant x.n\ and \ k>x.key_i$
	 	\li \Then
	 			$i=i+1$
	 		\End
	 	\li \If\ $x\leqslant x.n\ and\ k==x.key_i$
	 	\li \Then
	 			$\Return (x,i)$
	 		\End
	 	\li \ElseIf $x.leaf$
	 	\li    \Then \Return $NIL$
	 	    \End
	 	\li \Else 
	 	\li \Then$\proc{DISK\-READ}(x,c_i)$
	 	\li  \qquad\quad  \Return $\proc{B-TREE-SESRCH}(x.c_i,k)$
	 	    \End
	   \end{codebox}
  	\section{B-Tree的插入}
  	往B树插入元素的操作永远在B树的叶子结点上进行. 一个叶子根据B树的定义,除根结点外,每个结点的元素个数介于$t-1$和$2t-1$之间,所以插入元素的时候要考虑维护B树的性质.
  	
  	插入过程分为三种情况讨论:
  	\begin{enumerate}
  		\item 初始化情况,直接在根结点插入
  		\item 直接插入叶子结点后,满足B树的性质
  		\item 插入叶子结点不满足B树的性质,此时需要对叶子结点进行分裂
  	\end{enumerate}
  
    情况1比较简单,可以直接进行. 情况2和3都是递归的过程. 在向下搜索的过程中,为了避免回溯的发生,不能等到找到需要插入的结点才进行分裂,而是在递归向下的过程中,分裂沿途的每一个满结点.采取这样的操作后,如果是情况2,可以直接插入;如果是情况3,那么可以保证分裂叶子结点是,父结点不是满结点,不必回溯.分裂结点是使B树增高的唯一途径,而且一次只能增加一个高度.
    
    辅助过程$\proc{B-TREE-INSERT-NOFULL}$,决定了递归下降的方向和插入的位置,参数$x$表示关键字插入的结点,$k$表示关键字. 分裂过程$\proc{B-TREE-SPLIT-CHILD}(x,i)$表示分裂结点$x$的第$i$个孩子.$\proc{B-TREE-INSERTT}(T,k)$表示插入元素$k$
  	\begin{codebox}
  		\Procname{$\proc{B-TREE-SPLIT-CHILD}$(x,i)} 
  		\li $z=\proc{NEW\_NODE}()$
  		\li $y=x.c_i$
  		\li $z.leaf=x.leaf$
  		\li $z.n=t-1$
  		\li \For $z=1$ \To $t-1$
  		\li \Then $z.key_j=y.key_{j+t}$
  			\End
  		\li \If not $y.leaf$
  		\li \For $j=1$ \To $t$
  		\li \Then $z.c_i=y.c_{j+t}$
  			\End
  		\li $y.n=t-1$
  		\li \For $j=x.n+1$ \Downto $i+1$
  		\li \Then $x.c_{i+1}=x.c_{i}$
  			\End
  		\li $x.c_{i+1}=z$
  		\li \For $j=x.n$ \Downto $i$
  		\li \Then $x.key_{j+1}=x.key_j$
  			\End
  		\li $x.key_i=y.key_t$
  		\li $x.n=x.n+1$
  		\li $\proc{DISK-WRITE}(x)$
  		\li $\proc{DISK-WRITE}(y)$
  		\li $\proc{DISK-WRITE}(z)$
  	\end{codebox}
    \begin{codebox}
    	\Procname{$\proc{B-TREE-INSERTT-NOFULL}$(x,k)}
    	\li $i=x.n$
    	\li \If $x.leaf$
    	\li \Then \While$i\geqslant 1\ and\ k<x.key_i$
    	\li       \Then $x.key_{i+1}=x.key_{i}$
    	\li             $i=i-1$
    			  \End
    	\li       $x.key_{i+1}=k$
    	\li       $x.n=x.n+1$
    	\li       $\proc{DISK-WRITE(x)}$
    		\End
    	\li \Else \While $i\geqslant 1\ and\ k<x.key_{i}$
    	\li \Then \Then $i=i-1$
    			  \End
    	\li       $i=i+1$
    	\li 	  $\proc{DISK-READ}(x.c_i)$
    	\li		  \If $x.c_i.n==2t-1$
    	\li       \Then $\proc{B-TREE-SPLIT-CHILD}(x,i)$
    	\li             \If $k>x.key_i$
    	\li 		    \Then $i=i+1$
    					\End
    			  \End
    	\li       $\proc{B-TREE-INSERT-NOFULL}(x.c_i,k)$
    \end{codebox}
	\begin{codebox}
		\Procname{$\proc{B-TREE-INSERTT}$(T,k)}
		\li $r=T.root$
		\li \If $r.n==2t-1$
		\li \Then $s=\proc{NEW-NODE}()$
		\li 	  $T.root=s$
		\li 	  $s.leaf=FALSE$
		\li    	  $s.n=0$
		\li       $s.c_1=r$
		\li       $\proc{B-TREE-SPLIT-CHILD}(s,1)$
		\li       $\proc{B-TREE-INSERT-NOFULL}(s,k)$
			\End 
		\li \Else $\proc{B-TREE-INSERT-NONFULL}(r,k)$
	\end{codebox}
	\section{B-TREE的删除}
	B树的删除分为3个主要情况:
	
	\begin{enumerate}
		\item 关键字$k$在结点$x$中,$x$是叶结点,直接在$x$中删除$k$
		\item 关键字$k$在结点$x$中,$x$是内部结点,操作如下:
		\begin{enumerate}
			\item 结点$x$前于$k$的子结点$y$至少含有$t$个关键字,找出$k$在以$y$为根的子树中的前驱$k'$,\textbf{递归地}删除$k'$,并在$x$中用$k'$代替$k$.
			\item 对称的情况,如果$y$中少于$t-1$个结点,检查结点$x$中后于$k$的子结点$z$.
			如果$z$有$t$个关键字,则找出$k$在以$z$为根的子树中的后继$k'$.\textbf{递归}地删除$k'$,并在$x$中用$k'$代替$k$;
			\item $y$和$z$都只是有$t-1$个结点,则把$k$和$z$合并进入结点$y$,此时的$x$失去了$k$和指向$z$的指针,且$y$包含$t-1$个关键字.释放$z$,\textbf{递归}地从$y$中删除$k$
		\end{enumerate}
		\item 关键字$k$不在内部结点$x$中,则必包含于$x.c_i$中.如果$x.c_i$只有$t-1$个关键字,那么执行这两步保证下降到一个至少包含$t$个关键字的结点.
		
		\begin{enumerate}
			\item 如果$x.c_i$只有$t-1$个关键字,但它的兄弟至少有$t$个关键字,则把$x$中的某个关键字下降到$x.c_i$中,讲$x.c_i$的相邻左兄弟或右兄弟的一个关键字升至$x$,讲该兄弟中相应的孩子指针移到$x.c_i$中,使$x.c_i$增加一个关键字.
			\item 如果$x.c_i$以及$x.c_i$的所有兄弟结点都只有$t-1$个关键字,则把$x.c_i$与一个兄弟合并,即把$x$的一个关键字移到新合并的结点,使之成为该结点的中间关键字.
		\end{enumerate}
	\end{enumerate}
	几个函数的说明:
	\begin{enumerate}
		\item $\proc{MERGE-NODE}(x,i,y,z)$表示把内结点$x$的第$i$个孩子$y$和第$i+1$个孩子$z$合并,形成一个新结点$y$
		\item $\proc{B-TREE-PREDECESSOR}(y)$表示某结点在前驱子树根是$y$的情况下,查找在$y$中的递归意义上的前驱结点.使用迭代就行.
		\item $\proc{B-TREE-SUCCESSOR}(z)$表示某结点在后继子树根是$z$的情况下,查找在$z$中递归意义上的后继结点.
		\item $\proc{B-TREE-SHIFT-TO-RIGHT-CHILD}(x,i,y,z)$把$x$的第$i$左孩子$y$的一个结点转移给$y$的右兄弟$z$
		\item $\proc{B-TREE-SHIFT-TO-LEFT-CHILD}(x,i,y,z)$把$x$的第$i+1$右孩子$z$的一个结点转移给$z$的左兄弟$y$
		\item $\proc{B-TREE-DELETE-NONONE}(y,k)$删除内部结点$y$中的元素$k$
		\item $\proc{DISK-WRITE}(x)$把$x$的数据写入磁盘
		\item $\proc{FREE-NODE}(x)$释放$x$占用的空间
	\end{enumerate}
	\begin{codebox}
		\Procname{$\proc{MERGE-NODE}$(x,i,y,z)}
		\li $y.n=2t-1$
		\li \For $j=t+1$\To $2t-1$
		\li \Then $y.key_j=z.key_{j-1}$
			\End
		\li $y.key_t=x.key_i$
		\li \If $not\ y.leaf$
		\li \Then \For $j=t+1$\To$2t-1$
		\li 	  \Then $y.c_j=z.c_{j-t}$
		          \End
		    \End
		\li \For $j=i+1$ \To $x.n$
		\li \Then $x.c_j=x.c_{j+1}$
			\End 
		\li $x.n=x,n-1$
		\li $\proc{FREE-NODE}(z)$
		\li $\proc{DISK-WRITE}(x)$
		\li $\proc{DISK-WRITE}(y)$
		\li $\proc{DISK-WRITE}(z)$
	\end{codebox}
	\begin{codebox}
		\Procname{$\proc{B-TREE-DELETE-NONODE}$(x,k)}
		\li $i=1$
		\li \If $x.leaf$  \qquad\Comment Case 1
		\li \Then \While $i\leq x.n$ and $k>x.key_i$
	    \li       \Then $i=i+1$
	              \End
	    \li       \If $k==x.key_i$
	    \li       \Then \For $j=i+1$ \To $x.n$
	    \li 			\Then $x.key_{j-1}=x.key_j$
	    			    \End
	    \li             $x.n=x.n-1$
	    \li             $\proc{DISK-WRITE(x)}$
	    \li       \Else $error\ "key\ does\ not\ exist"$
	    		  \End
	    \li \Else \While $i<=x.n$ and $k>x.key_i$
	    \li				\Then  $i=i+1$
	    				\End
	    \li       $\proc{DISK-READ}(x.c_{i+1})$
	    \li 	  $y=x.c_i$
	    \li       \If $i\leqslant x.n$
	    \li       \Then $\proc{DISK-READ}(x.c_{i+1})$   
	    \li             $z=x.c_{i+1}$
	    		  \End 
	    \li       \If $k==x.key_i$ \qquad\Comment Case 2
	 	\li       \Then  \If $y.n>t-1$ \qquad\Comment Case 2a
	 	\li              \Then $k'=\proc{B-SEARCH-PREDECESSOR(y)}$
	 	\li 				   $\proc{B-TREE-DELETE-NONONE}(y,k')$
	 	\li                    $x.key_i=k'$
	 	\li              \ElseIf $z.n>t-1$\qquad\Comment Case 2b
	 	\li				 \Then $k'=\proc{B-TREE-SUCCESSOR}(z)$
	 	\li                    $\proc{B-TREE-DELETE-NONONE}(z,k')$			
	 	\li                    $x.key_i=k'$	 
	    		         \Else $\proc{B-TREE-MERGE-CHILD}(x,i,y,z)$\qquad\Comment Case 2c
		\li					   $\proc{B-TREE-DELETE-NONONE}(y,k)$    		         
	    		  \End
	    \li       \Else \qquad\Comment Case 3
	    \li             \If $i>1$
	    \li				\Then $\proc{DISK-READ}(x.c_{i-1})$
	    \li                   $p=x.c_{i-1}$
	   					\End
	    \li             \If	  $y.n==t-1$
	    \li 			\Then \If $i>1$ and $p.n>t-1$ \qquad\Comment Case 3a
	    \li                   \Then $\proc{B-TREE-SHIFT-TO-RIGHT-CHILD}(x,i-1,p,y)$
	                          \ElseIf $i<=x.n$ and $z.n>t-1$ \qquad\Comment Case 3a
        \li	                  \Then    $\proc{B-TREE-SHIFT-TO-LEFT-CHILD}(x,i,y,z)$
        \li                   \ElseIf $i>1$ \qquad\Comment Case 3b
        \li                   \Then   $\proc{MERGE-NODE}(x,i-1,p,y)$
        \li                           $y=p$
        \li                   \Else $\proc{MERGE-NODE}(x,i,y,z)$ \qquad\Comment Case 3b
        					  \End	
        					  \End
        \li		        $\proc{B-TREE-DELETE-NONONE}(y,k)$
	\end{codebox}
	\begin{codebox}
		\Procname{$\proc{B-TREE-PREDECESSOR}$(y)}
		\li $x=y$
		\li $i=x.n$
		\li \While not $x.leaf$
		\li \Then $\proc{DISK-READ}(x.c_{i+1})$
		\li       $x=x.c_{i+1}$
		\li       $i=x.n$
			\End
		\li \Return $x.key_i$
	\end{codebox}
	\begin{codebox}
		\Procname{$\proc{B-TREE-SUCCESSOR}$(z)}
		\li $x=z$
		\li \While not $x.leaf$
		\li \Then $\proc{DISK-READ}(x.c_{1})$
		\li       $x=x.c_1$
		\End
		\li \Return $x.key_1$
	\end{codebox}
	\begin{codebox}
		\Procname{$\proc{B-TREE-SHIFT-TO-RIGHT-CHILD}$(x,i,y,z)}
		\li $z.n=z.n+1$
		\li $j=z.n$
		\li \While $j>1$
		\li \Then $z.key_j=z.key_{j-1}$
		\li       $j=j-1$
				  \End
		\li $z.key_1=x.key_i$
		\li $x.key_i=y.key_{y.n}$
		\li \If not $z.leaf$
		\li \Then $j=z.n$
		\li       \While $j>0$
		\li       \Then $z.c_{j+1}=z.c_j$
		\li       $j=j-1$
				  \End
		\li       $z.c_1=y.c_{y.n+1}$
		     \End
		\li $y.n=y.n-1$
		\li $\proc{DISK-WRITE}(x)$
		\li $\proc{DISK-WRITE}(y)$
		\li $\proc{DISK-WRITE}(z)$
	\end{codebox}
	\begin{codebox}
		\Procname{$\proc{B-TREE-SHIFT-TO-LEFT-CHILD}$(x,i,y,z)}
		\li $y.n=y.n+1$
		\li $y.key_{y.n}=x.key_i$
		\li $x.key_i=z.key_1$
		\li $z.n=z.n-1$
		\li $j=1$
		\li \While $j\leqslant z.n$
		\li \Then $z.key_j=z.key_{j+1}$
		\li       $j=j+1$
			\End
		\li \If not $z.leaf$
		\li \Then $y.c_{y.n+1}=z.c_1$
		\li       $j=1$
		\li       \While $j\leqslant z.n+1$
		\li       \Then $z.c_j=z.c_{j+1}$
		\li             $j=j+1$
				  \End
		    \End
		\li $\proc{DISK-WRITE}(x)$
		\li $\proc{DISK-WRITE}(y)$
		\li $\proc{DISK-WRITE}(z)$
	\end{codebox}
 	\section{总结}
	\section{实验测试}
\end{document}
